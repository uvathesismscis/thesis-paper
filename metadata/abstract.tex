\begin{abstract}
Suboptimal Indoor Air Quality (IAQ) have been identified to pose health risks and affect the cognitive performance of occupants within indoor environments. The study investigates occupants' awareness and perceptions of IAQ, facilitating occupants adoption of preventive measures. Employing a user-centered mixed-methods approach, the research employs Post-Occupancy Evaluation (POE) methods to elicit occupants' understanding of indoor air quality and utilizes sensory monitors to collect and measure common pollutants in specific indoor environments. These insights guide the development of 'Phair', an interactive data physicalization that visualizes real-time air quality data which was subsequently evaluated and usability-tested with occupants. The results of the user study reveal a lack of awareness among users regarding indoor air quality, while evaluations of the prototype demonstrate that physical hardware displaying real-time data can effectively increase occupants' awareness.

\end{abstract}