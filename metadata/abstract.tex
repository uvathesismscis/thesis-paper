\begin{abstract}
Suboptimal Indoor Air Quality (IAQ) poses health risks and affects the cognitive performance of occupants within indoor environments. Using a user-centered mixed-methods approach, this study employs Post-Occupancy Evaluation (POE) methods to elicit occupants' understanding of indoor air quality. It utilizes sensory monitors to collect and measure common pollutants in specific indoor environments. These insights guide the development of 'Bluebird,' an interactive data physicalization that visualizes real-time air quality data, which was subsequently evaluated and usability-tested with occupants. The results reveal a need for more awareness among users regarding indoor air quality and that CO2 concentrations regularly exceed the optimal threshold within meeting rooms. Evaluations of the data physicalization indicate that real-time environmental data can effectively increase occupants' awareness and help them take preventive action.
\end{abstract}