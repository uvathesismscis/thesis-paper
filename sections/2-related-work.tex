\section{Related Work}
\label{sec:related_work}
% Your work needs to be grounded and compared to earlier work and the state-of-the-art. Start the section with announcing the research gap and also end with the research gap. Consider using hypotheses. 

This section provides and overview of studies conducted on the involvement of Human-Computer Interaction (HCI) in studying Built Environments. It begins by introducing the overarching concept of Human-Building Interaction (HBI) and subsequently narrows its focus to Indoor Air Quality (IAQ) and Post-occupancy Evaluation (POE) for the scope of this research. Finally, it presents key findings from previous approaches to mapping and encoding sensory data in a new area of research called Data Physicalization.

\subsection{Human-Building Interaction}

Buildings increasingly incorporate new forms of digital interaction \cite{pulsipher_towards_2023, margariti_understanding_2023}, which means new inherent connections between 'people', 'built environments', and 'computing' research in an area called Human-Building Interaction (HBI) \cite{alavi_introduction_2019, taherkhani_human-building_2023}. This research area is dedicated to exploring the design of built environments that may incorporate computing to varying degrees \cite{sowles_introducing_2021}.  A logical extension where indoor spaces are increasingly rerofitted with Ubiquitous Computing (Ubicomp) \cite{weiser_computer_1999,bell_yesterdays_2007} sensing devices. Understanding how people use different spaces in a building through computing can inform design interventions aimed at improving the utility of the space and well-being of occupants. \cite{verma_studying_2017}. 

Current research into architecture and built environments indicate that a significant portion of the data collected by these computing devices are not necessarily transparent or comprehesible to occupants \cite{schnadelbach_adaptive_2019}, and indoor spaces are designined without much thought of placing of computing devices integrated within the environment \cite{johansen_temporal_2019, kirsh_architects_2019}. Numerous approaches to study computing devices within indoor spaces and interactivity with occupants have been explored in prior research \cite{bader_windowwall_2019, rogers_ambient_2010}.


\subsection{Comfort within buildings}

Comfort is achieved in interaction with the environment and is represented in four respective dimensions; thermal, respiratory, visual, and acoustic \cite{comfort}. Comfort can be studied and designed as an interactive experience with the built environment itself \cite{environment}. Indoor Environmental Quality (IEQ) indexes serve as metrics for assessing comfort, with Post-Occupancy Evaluation (POE) being employed to gauge occupants' perceived comfort. In current scenarios, technology is typically retrofitted onto a new or existing building and users indicate a perceived lack of control and engagement with these systems, primarily because many automated buildings operate based on arbitrarily set parameters, and data these sensors gather are often invisible to end users.

Focus on broader field of indoor enviroment qualities. Problem is that users have limited percieved control over automated systems and  Also buildings are not adaptible to users subjective needs and technology is typically retrofitted onto existing building allowing limited interaction by occupants taking preventive action on their comfort \cite{alavi_comfort_2017}.


\subsection{Indoor Air Quality}

Focus on health, cognitive of indoor air quality. Mention paper Hamed about office meetings. Mension paper about subjective indoor air measurements.


\subsection{Data Physicalization}

\subsection{Persuasive Technology}