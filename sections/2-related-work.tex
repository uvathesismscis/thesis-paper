\section{Related Work}
\label{sec:related_work}
Given that the focus of this research is studying Human-Computer Interaction (HCI) within Built Environments (BE), this research draws from various related work, including the subfield of Human-Building Interaction (HBI) (see \hyperref[sec:hbi]{Section \ref*{sec:hbi}}) and subsequently narrows its focus on Comfort withing Buildings (see \hyperref[sec:poe]{Section \ref*{sec:poe}}) and indoor air quality (IAQ) (see \hyperref[sec:iaq]{Section \ref*{sec:iaq}}) for the scope of this research. Furthermore, it examines notable findings from previous approaches to map and encode sensory data in a new area of research called Data Physicalization (DataPhys) (see \hyperref[sec:phys]{Section \ref*{sec:phys}}).

\subsection{Human-Building Interaction}
\label{sec:hbi}

Buildings increasingly incorporate new forms of digital interaction \cite{pulsipher_towards_2023, margariti_understanding_2023}, which means new inherent connections between 'people', 'built environments' and 'computing' research in an area called Human-Building Interaction (HBI) \cite{alavi_introduction_2019, taherkhani_human-building_2023}. This research area is dedicated to exploring the design of built environments that may incorporate computing to varying degrees \cite{sowles_introducing_2021}. A logical extension where indoor spaces are increasingly retrofitted with sensing devices \cite{pulsipher_towards_2023}. Understanding how people use different spaces in a building through computing can inform design interventions aimed at improving the utility of the space and the well-being of occupants. \cite{verma_studying_2017}. Current research into architecture and built environments indicates that a significant portion of the data collected by these computing devices is not necessarily transparent or comprehensible to occupants \cite{schnadelbach_adaptive_2019}. Additionally, architects and interior designers often integrate computing devices into indoor spaces without much thought  \cite{johansen_temporal_2019, kirsh_architects_2019}, leaving users with a perceived lack of control over their indoor comfort. 

\subsection{Comfort within buildings}
\label{sec:poe}

Indoor occupant comfort is interaction with the environment and manifests in four respective dimensions: thermal, respiratory, visual, and acoustic \cite{alavi_comfort_2017}. Indoor Environmental Quality (IEQ) \cite{kulshreshtha_indoor_2024} indexes serve as metrics for assessing the aforementioned properties of comfort within indoor environments with Post-Occupancy Evaluation (POE) \cite{elsayed_post-occupancy_2023} and Perceived Environmental Qualities (PEQ) \cite{son_perceived_2023} methods are employed to gauge occupants' perceived comfort \cite{boissonneault_concepts_2023}. 

Studies on indoor environments focus on 'static' IEQ conditions using sensors to sense environmental conditions based on the buildings' physical characteristics to meet various recommended standards such as ASHRAE 62.1 \footnote{https://www.ashrae.org/technical-resources/bookstore/standards-62-1-62-2} and ISO 16814 \footnote{https://www.iso.org/standard/42720.html}. Discrepancies between measured IEQ conditions and occupants' perceptions have also been reported in studies. For instance, research indicates that occupants generally have a low awareness of Indoor Environmental Quality (IEQ). While occupants perceive the environment as 'satisfactory', actual sensory measurements within the environment reveal quality levels below recommended standards. \cite{son_perceived_2023}. Recent studies have shifted their focus towards the active role occupants play within the built environment, viewing their behavior within a building as akin to a 'living ecology' \cite{langevin_quantifying_2016} rather than perceiving comfort solely as 'static' properties of the building itself. 


\subsection{Indoor air quality}
\label{sec:iaq}

A suboptimal indoor environment is reportedly associated with health-related problems such as headaches, throat irritation, and asthma \cite{klepeis_national_2001} as well as a decrease in cognitive functions such as tiredness, effects on performance and productivity and a lack of focus \cite{wang_how_2021} \cite{du_indoor_2020}. A phenomenon referred to as the Sick Building Syndrome (SBS) \cite{gawande_indoor_2020, passarelli_sick_2009}. Many of these symptoms are primarily associated with respiratory comfort and are closely tied to indoor air quality (IAQ) concerns \cite{kim_analyzing_2019}. Effective ventilation strategies
have significantly alleviated SBS symptoms \cite{gawande_indoor_2020}.

The advancements of real-time IAQ monitoring systems leveraging Internet of Things (IoT) sensor technology have facilitated progress in both the measurement of IAQ and the implementation of interventions aimed at enhancing it \cite{pantelic_transformational_2022}. Indications of poor air quality are gathered by measuring common pollutants with a focus on molds and allergens (humidity), volatile organic compounds (VOC), and carbon dioxide (CO2) \cite{klepeis_national_2001} where occupant behavior and the number of occupants within an indoor space has a specific negative effect on CO2 levels \cite{fromme_indoor_2023}. Building occupants' behaviors influence these indoor climate factors, which require special attention when assessing IEQ conditions and determining the presence of adequate ventilation. \cite{du_indoor_2020}. It is crucial to recognize that when occupants experience symptoms, it signifies that a suboptimal air quality situation has already occurred.

The existing literature on IAQ offers quantifiable and validated methods for measuring IAQ through sensory data. It underscores the complexities associated with IAQ and emphasizes the significance of developing solutions to ensure occupants receive adequate ventilation.


\subsection{Data Physicalization}
\label{sec:phys}

The research domain known as data physicalization \cite{alexander_data_2019, jansen_opportunities_2015} has emerged as a notable area of study, emphasizing the creation of physical data visualizations, making the invisible tangible and interactable by encoding data in physical artifacts \cite{ranasinghe_encoding_2023}. This shift from focusing on individual artifacts in broader environmental context facilitates the physical embodiment of computing \cite{dragicevic_data_2020}. Data physicalization has the potential to positively influence the perception and exploration of data \cite{jansen_opportunities_2015, wang_emotional_2019, stusak_evaluating_2015}, presenting distinct advantages over traditional 'screen-focused' data representations, such as 2D canvas displays (e.g, digital web-based dashboards, screens within the room) \cite{hornecker_design_2023, jansen_evaluating_2013}, particularly in the context of indoor air quality (IAQ) where a 'physical data visualization' serves as a fitting metaphor for rendering 'invisible' indoor air. These tangible artifacts usually come in the form of ubiquitous computing (ubicomp) \cite{bell_yesterdays_2007} device that seamlessly blends into the environment, essentially making the computing devices 'disappear' \cite{weiser_computer_1999}. 

These devices are frequently employed as persuasive technology, strategically designed to nudge individuals toward gentle
behavior change leveraging the emerging notion of pervasive sensing to subtly enhance users' awareness regarding the impacts of their decisions \cite{bader_windowwall_2019, rogers_ambient_2010}. This method of persuasive design serves as a powerful tool in calmly extending users' awareness, helping users understand gathered data and the consequences of their actions, and gaining insight into their behavior \cite{bae_making_2022}. A systematic analysis of over 60 representative data physicalization papers \cite{sauve_physecology_2022} shows that only numerous ($f$=3) approaches to studying computing devices within indoor spaces and interactivity with occupants have been explored in prior research especially with a focus on indoor air quality to create awareness and nudge occupants to a desired behavior and rarely on a large-scale architectural intervention which is the focus of this research.

This framework of data physicalization and persuasive technology establishes the theoretical foundation for the creation (prototyping) and the evaluation (usability testing) of the design solution within this research.