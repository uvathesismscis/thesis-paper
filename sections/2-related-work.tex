\section{Related Work}
\label{sec:related_work}
% Your work needs to be grounded and compared to earlier work and the state-of-the-art. Start the section with announcing the research gap and also end with the research gap. Consider using hypotheses. 

This section provides and overview of studies conducted on the involvement of Human-Computer Interaction (HCI) in studying Built Environments. It begins by introducing the overarching concept of Human-Building Interaction (HBI) and subsequently narrows its focus to Indoor Air Quality (IAQ) and Post-occupancy Evaluation (POE) for the scope of this research. Finally, it presents key findings from previous approaches to mapping and encoding sensory data in a new area of research called Data Physicalization.

\subsection{Human-Building Interaction}

Buildings increasingly incorporate new forms of digital interaction \cite{pulsipher_towards_2023, margariti_understanding_2023}, which means new inherent connections between 'people', 'built environments', and 'computing' research in an area called Human-Building Interaction (HBI) \cite{alavi_introduction_2019, taherkhani_human-building_2023}. This research area is dedicated to exploring the design of built environments that may incorporate computing to varying degrees \cite{sowles_introducing_2021}.  A logical extension where indoor spaces are increasingly rerofitted with sensing devices \cite{pulsipher_towards_2023}. Understanding how people use different spaces in a building through computing can inform design interventions aimed at improving the utility of the space and well-being of occupants. \cite{verma_studying_2017}. 

Current research into architecture and built environments indicate that a significant portion of the data collected by these computing devices are not necessarily transparent or comprehesible to occupants \cite{schnadelbach_adaptive_2019}, and indoor spaces are designined without much thought of placing of computing devices integrated within the environment \cite{johansen_temporal_2019, kirsh_architects_2019} leaving users with a perceived lack of control over their indoor comfort. 

\subsection{Comfort within buildings}

Comfort is achieved in interaction with the environment and is represented in four respective dimensions; thermal, respiratory, visual, and acoustic \cite{alavi_comfort_2017}. Indoor Environmental Quality (IEQ) \cite{kulshreshtha_indoor_2024} indexes serve as metrics for assessing the afore mentioned properties of comfort within indoor environments with Post-Occupancy Evaluation (POE) \cite{elsayed_post-occupancy_2023} and Perceived Environmental Qualities (PEQ) \cite{son_perceived_2023} methods being employed to gauge occupants' perceived comfort \cite{boissonneault_concepts_2023}. 

Studies on indoor environments focus on 'static' IEQ conditions using sensors to sense environmental conditions based on the buildings' physical characteristics to meet various recommended standards such as ASHREA 62.1 \footnote{https://www.ashrae.org/technical-resources/bookstore/standards-62-1-62-2}, ISO 16814 \footnote{https://www.iso.org/standard/42720.html}. Descrepancies between measured IEQ conditions and occupants perceptions have also been reported in studies. For example, studies have shown occupants general awareness of IEQ is low, occupants perceived the environment as 'satisfactory' but actual sensory measurement within the environment showed quality levels below the recommended standards \cite{son_perceived_2023}. Recent studies are more focussed on the active role occupants play within the built environment and their behavior considering the activity within a building as a 'living ecology' itself \cite{langevin_quantifying_2016}. 


\subsection{Indoor Air Quality}

A suboptiomal indoor environment has reportedly been associated with health-related problems such as headaches, throat irritation and asthma \cite{klepeis_national_2001} as well as a decrease in cognitive functions such as tiredness, effects on performance and productivity and a lack of focus \cite{wang_how_2021} \cite{du_indoor_2020}. A phenomenon often referred to as the Sick Building Syndrome (SBS) \cite{gawande_indoor_2020, passarelli_sick_2009}. Most of these symptoms are predominantly associated with respitatory comfort and related to the Indoor Air Quality (IAQ) \cite{kim_analyzing_2019}. Proper ventilation can drastically reduce SBS symptoms \cite{gawande_indoor_2020}.

The evolution of real-time IAQ monitoring systems leveraging Internet of Things (IoT) sensor technology has facilitated progress in both the measurement of IAQ and the implementation of interventions aimed at enhancing it \cite{pantelic_transformational_2022}. Indications of poor air quality are gathered by measuring common pollutants with a focus on molds and allergens (humidity), volatile organic compounds (VOC) and carbon dioxide (CO2) \cite{klepeis_national_2001} where occupant behaviour and the number of occupants within indoor space have a specific negitative effect on CO2 levels \cite{fromme_indoor_2023}. These indoor climate factors are related to the building occupants’ behaviors and need special attention to be considered in assessing the IEQ conditions and determining if adequate ventilation is present \cite{du_indoor_2020}. When occupants experience symptoms this means that a suboptimal air quality situation has already occurred. With the ability to use real-time sensor readings it's 

The existing literature on indoor air quality provides quantifiable and tested methods to measure indoor air quality using sensory data, stresses the complications associated with IAQ and the importance of creating solutions to providing adequate ventilation to occupants. 


\subsection{Data Physicalization}

The research domain known as data physicalization \cite{alexander_data_2019} has emerged as a notable area of study, emphasizing the creation of physical data visualizations making the invisible tangible and interactible by encoding data in physical artifacts \cite{ranasinghe_encoding_2023}. This shift from focusing on individual artifacts to a broader environmental context facilitates the physical embodiment of computing. Data physicalization has the potential to positively influence the perception and exploration of data \cite{jansen_opportunities_2015}, presenting distinct advantages over traditional 'screen-focused' data representations, such as 2D canvas displays (dashboards). \cite{hornecker_design_2023} especially with the focus on IAQ where a 'physical data visualization' is a fitting metaphor for visualizing 'invisible indoor air'.

These tangible artefacts usually come in the form of a ubiquitous computing (ubicomp) \cite{bell_yesterdays_2007} device that seamlessly blends into the environment, essentially making the computing devices 'disappear' \cite{weiser_computer_1999}. These devices are frequently employed as persuasive technology, strategically designed to gently nudge individuals towards behavior change leveraging the emerging notion of pervasive sensing to subtly enhance users' awareness regarding the impacts of their decisions \cite{bader_windowwall_2019, rogers_ambient_2010}. This method of persuasive design serves as a powerful tool in calmly extending users' awareness, helping users understand gathered data, the consequences of their actions, and gaining insight into their behavior \cite{bae_making_2022}. Numerous approaches to study computing devices within indoor spaces and interactivity with occupants have been explored in prior research \cite{sauve_physecology_2022} to nudge occupants to a desired behaviour but limited design solutions have been developed and explored to focus specificaly on IAQ awareness. 



This framework of data physicalization and persuasive technology establishes the theoretical foundation for the creation (prototyping) and the evaluation (usability testing) of the design solution within this research.