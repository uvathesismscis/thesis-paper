\section{Methodology}
\label{sec:methodology}

To answer the research question, this study uses a human-centered approach (often referred to as Design Thinking) commonly found in Human-Computer Interaction studies consisting of four phases: 1) user studies to \textit{understand} the user, 2) \textit{data collection} methods, and analysis of the situation as field trials, 3) \textit{ideation} and experimental design of a prototype and, 4) \textit{evaluation} and usability testing of the prototype \cite{jonathan_lazar_research_2017, zimmerman_research_2007}.

This mixed-methods approach (both qualitative and quantitative) helps understand occupants' needs and informs the design and the technical set-up of the prototype evaluating the effectiveness by focusing on the user's needs from the start of the study \cite{rogers_moving_2017, experience_ux_2024}. 
In the first phase, an online questionnaire was conducted to \textit{understand} occupants' awareness of indoor air quality (see \hyperref[sec:questionnaire]{Section \ref*{sec:questionnaire}}). For the second phase, IAQ monitors are installed within meeting rooms for \textit{data collection} and to gain insight into environmental data (see \hyperref[sec:monitoring]{Section \ref*{sec:monitoring}}) which in turn informed phase three the \textit{ideation} and creation of the prototype (see \hyperref[sec:ideation]{Section \ref*{sec:ideation}}). In the last phase participants \textit{evaluated} the prototype and performed usability tests (see \hyperref[sec:evaluation]{Section \ref*{sec:evaluation}}). Before setting up the survey, creating the prototype, and conducting the evaluation interviews, the investigation focused on the concepts of indoor air quality monitoring and data physicalization through a literature review and desk research.

\subsection{Case study building}

This study is conducted in association with the Digital Interactions Lab \footnote{https://uva-dilab.com/} and will utilize the recently opened Lab42 \footnote{https://lab42.uva.nl/} building at the UvA Amsterdam Science Park \footnote{https://www.amsterdamsciencepark.nl/} as its primary case study location (see \hyperref[appendix:building]{Appendix \ref*{appendix:building}}). Lab42 is an energy-neutral, flexible, and adaptable faculty building that facilitates collaborations among students, researchers, and businesses \cite{benthem_2022}. Since most of the space within the building are designated as informal learning spaces and another large part of the building is designed as meeting rooms (see \hyperref[appendix:meetings]{Appendix \ref*{appendix:meetings}}) occupants are most likely to experience the effects of poor indoor air quality within the meeting rooms due to the rather dense and crowded indoor environment. Lab42 is an example of a smart building or living lab where sensing devices are retrofitted throughout the building to automatically adjust lighting, temperature, and the focus of this research regulating air \cite{architects_lab42_2022}. This already provides a base of environmental data that can be used and extended for further analysis. 

\newpage

\subsection{Questionnaire survey}
\label{sec:questionnaire}

To understand and collect occupants' subjective awareness and satisfaction with IAQ, a structured survey gathered quantitative data within the building as a form of Post Occupancy Evaluation (POE) (see \hyperref[appendix:survey]{Appendix \ref*{appendix:survey}}).


\subsubsection{Questions}
The questions draws from two POE studies with a focus on indoor air quality \cite{silva_post-occupancy_2017, son_perceived_2023} usng standardized questions (e.g., Q-bank) and scales (e.g., Likert scale) similar to Customer Satisfaction (CSAT) surveys. The survey consisted of a total of nine questions (5 multiple choice, 3 Likert scales, 1 not mandatory open question) consisting of questions about:

\begin{enumerate}
  \item \textit{Activity and occupancy:} the rough location the occupant is within the building, how often the occupants use the building for various activities, and how they would describe the occupancy in their current space.
  \item \textit{Awareness and satisfaction:} how aware the occupant is of the current air quality in the space, how the occupant perceives the air quality in the current space, and how satisfied the occupant is with the air quality in the current space.
  \item \textit{Health and cognitive symptoms :} if the occupant experiences any health or cognitive symptoms based on the air quality in the current space.
\end{enumerate}


\subsubsection{Participants}
Handouts with QR Codes to the survey are distributed to occupants present at the informal learning spaces of the atrium, first, and second floors, using eligibility criteria based on demographic characteristics. There were no additional inclusion criteria besides the convenience sampling size of respondents being present in the case study building (sampling in context). Additionally, handouts were attached to the tables using stickers. All instances of participation were voluntary and conducted without remuneration. 

\subsubsection{Responses}
The survey was open for submission for eight weeks, from March 1 \textsuperscript{st} to April 31\textsuperscript{st}, 2024 which recorded 32 responses in total, of which after cleanup, a total of 29 ($n$=29) responses were included in the final dataset. The survey did not collect any personally identifiable information, such as age and gender, and it adhered to ethical considerations (e.g., consent forms). To improve the quality of the questionnaire on the initial version feedback was requested, after which the survey protocol was piloted before
distributing to participants (see \hyperref[appendix:experts]{Appendix \ref*{appendix:experts}}). 

\subsubsection{Data cleaning, preprocessing and analysis}
\label{sec:analysis}
Following the survey distribution, data analysis involved cleaning the data (e.g., removing non-consenting users) using Python's  Jupyter Notebook format \footnote{https://jupyter.org/} and visualizing insights with tools like Seaborn \footnote{https://seaborn.pydata.org/}.  This process included creating graphs and plots (e.g., boxplots of Likert scales) to gain an overview and understanding of the occupants' perspectives.

\newpage

\subsection{Air Quality Monitoring}
\label{sec:monitoring}

To gather data about the current air quality situation within the building, IAQ monitors were retrofitted to two specific meeting rooms (see \hyperref[appendix:floorplan]{Appendix \ref*{appendix:floorplan}}). This collected data further informed and provided the basis for inputting data into the data physicalization process.

\subsubsection{Technical set-up}

Monitors were deployed in meeting rooms regularly used by occupants to understand their behavior and perception in real-world corporate settings rather than controlled lab environments. These rooms are referred to as the small room (Room A), measuring 18 m\textsuperscript{2} and typically accommodating small meetings with seven seats, and the large room (Room B), measuring 48 m\textsuperscript{2} and suitable for larger meetings and seminars with fourteen seats. Two commercially available indoor climate data loggers were installed using 3D-printed mounting plates: an AirCheq Touch Aero \footnote{\url{https://airteq.eu/producten/touch-aero/}} in the smaller room and an Atal ATU-CT ClimaTrend \footnote{\url{https://www.atal.nl/atu-ct-climatrend-binnenklimaat-datalogger}} in the larger room. Both monitors utilize industry-standard sensors from manufacturers like Senseirion and SenseAir to measure common pollutants. They were mounted (between 80cm and 120cm from the ground) and calibrated (intervals and polling rates) according to the installation manuals provided by the manufacturers. The monitoring devices operated for four weeks, from April 1\textsuperscript{st} to April 30\textsuperscript{th}.

\subsubsection{Data logs}

The data gathered by the sensors provides insights into various standardized measurements related to common pollutants that affect IAQ, such as molds and allergens (humidity), volatile organic compounds (VOC), and carbon dioxide (CO$_{2}$) (see \hyperref[appendix:monitors]{Appendix \ref*{appendix:monitors}}). The data logs were cross-referenced with weekly schedules derived from the internal booking systems of the rooms. This alignment was based on timestamped data to synchronize sensor values with scheduled meetings. Data analysis followed a procedure similar to that described in (see \hyperref[{sec:analysis}]{Section \ref*{sec:analysis}}). The data underwent cleaning to remove entries primarily from non-opening hours and was then plotted and visualized to explore patterns and cross-referenced with meeting times.

\subsection{Ideation and requirements}
\label{sec:ideation}

As a starting point for creating a physical representation of the air quality data, the formative research is grounded in the growing interest in establishing theoretical and design foundations for \textit{data physicalisation} \cite{hornecker_design_2023, sauve_physecology_2022, bae_making_2022} on how to encode the properties and use a common design language \cite{ranasinghe_encoding_2023, sosa_data_2018} established by systematic reviews of data physicalization projects. The creation of a prototype (see \hyperref[fig:prototype-impressions]{Figure \ref*{fig:prototype-impressions}}) illustrates the technology to be and are often deployed to conduct lab experiments and case studies \cite{jonathan_lazar_research_2017}. The following definition describes the overall goal of the physicalization:

\AtBeginEnvironment{quote}{\setlength{\leftmargini}{10pt}}
\AtBeginEnvironment{quote}{\itshape}
\begin{quote}
a data-driven physical artifact whose geometry and material properties encode data that aims to augment a nearby audience’s understanding of data insights.
\end{quote}


The aim of the prototype is to gather knowledge about the intervention and human behavior around it, aligning with a pragmatist methodology in design research where methods adapt and evolve through interactions with participants. Before prototyping the design solution, the overall requirements and scope of the physicalization and case studies used for ideation are described. The final prototype is then described based on the encoded variables and design dimensions found in the literature (see \hyperref[sec:prototype_results]{Section \ref*{sec:prototype_results}}).

\begin{figure}[b]
    \centering
    \includegraphics[width=0.5\textwidth]{prototype_impressions.jpg}
    \caption{Impressions of the prototyping and ideations phase. A) Gateway device B) Testing servo motors C) Writing firmware code}
    \label{fig:prototype-impressions}
\end{figure}

\subsubsection{Concept requirements}

Based on this scope and the survey, the concept requirements ("r" for "requirements") of the design solution are described in further detail, prioritizing six overarching requirements ranked using the Moscow method:

\begin{enumerate}
    \renewcommand{\labelenumi}{R\arabic{enumi}:}
    \item \textbf{Visual Feedback:} The prototype must provide visual feedback through movement encoding environmental properties that represent air quality metrics, ensuring that users can easily interpret the information conveyed (must have).
    \item \textbf{Size and location:} The prototype must be designed to be installed within small to medium-sized rooms, with consideration for its dimensions and weight to ensure compatibility (must have).
    \item \textbf{Real-Time Data Integration:} The prototype should integrate real-time data from air quality monitors to dynamically adjust its behavior (should have).
    \item \textbf{Interactive sensing:} The prototype should be interactive in which occupants can interact with the prototype by walking near it providing a tactile experience (should have).
    \item \textbf{Material durability:} The prototype could use natural materials and be durable and resistant to environmental factors such as humidity and temperature fluctuations (could have).
    \item \textbf{Aesthetic Integration:} The prototype could seamlessly integrate with its surrounding environment, complementing interior design aesthetics, architectural features, and layouts of the room to enhance the overall ambiance (could have).
\end{enumerate}

\subsubsection{Concept ideation}
In developing the concept models, two existing datasets of academic and non-academic case studies served as comparative studies and for desk research, providing a starting point for ideation. First was the DataPhys gallery \footnote{http://dataphys.org/list/gallery/}, a collection of 372 entries classified as data physicalizations. The second was a combination of three state-of-the-art papers with systematic reviews of physicalization with combined examples of around 132 entries classified as data physicalization, which consisted of both academic and non-academic samples \cite{sauve_physecology_2022, anhalt_university_germany_design_2022, ranasinghe_encoding_2023}. Out of these, ten ($f$=10) samples of academic work were selected for further review based on the similarity with the before described requirements of which three ($f$=3) samples with a focus on the environmental property of air, but not necessarily air quality within indoor environments (see \hyperref[appendix:academic]{Appendix \ref*{appendix:academic}}). Additionally, twenty-four ($f$=24) samples of non-academic case studies were reviewed after desk research, which included work and prototypes from design studios and independent creators that informed the ideation phase (see \hyperref[appendix:nonacademic]{Appendix \ref*{appendix:nonacademic}}). The aim was to find samples and design solutions that focussed on ambient interaction and had similar social contexts. This analysis led to a better conceptualizing of the design based on the aforementioned requirements of the data physicalization. 

\subsubsection{Concept models}

Based on the user requirements and ideation three low-fidelity (lo-fi) concepts were further elaborated (see \hyperref[appendix:conceptdiagrams]{Appendix \ref*{appendix:conceptdiagrams}}) to choose one to develop in high fidelity (hi-fi) for user studies and evaluation. For this concepting, methods from the Communication and Multimedia Design (CMD) Methods Pack \footnote{https://cmdmethods.nl/} and Design Method Toolkit \footnote{https://toolkits.dss.cloud/design/} by the Digital Society School (DSS) were used. Concept selection was based on weighted physicalization criteria from the literature, a Harris profile for the lo-fi concepts, and feedback from researchers with HCI and data visualization backgrounds ($n$=6 researchers not involved in the project, see \hyperref[appendix:experts]{Appendix \ref*{appendix:experts}}). Also, technical limitations of the provided hardware (e.g., real-time data output of monitors, cost of hardware, availability of electronic components,) and limitations in the technical set-up of the building (e.g., space in the meeting rooms, not allowed to alter furniture) were considered as heuristic evaluation. Based on the aggregation of these criteria, the \textit{Bluebird} concept was chosen to be further developed into a high-fidelity prototype (see \hyperref[appendix:prototype]{Appendix \ref*{appendix:prototype}}).

\subsection{Evaluation}
\label{sec:evaluation}

A field-based evaluation approach within a meeting room of the Lab42 building was employed, utilizing methods from the Human-centered Design Kit by Ideo \footnote{https://www.designkit.org/methods.html} and the Delft Design Guide from Delft University of Technology (TU) \footnote{https://www.bispublishers.com/delft-design-guide-revised.html}. For evaluation criteria, the intentions and evaluating interview methodology described in the data physicalization design vocabulary \cite{jansen_evaluating_2013,ranasinghe_encoding_2023} are used as a baseline for evaluating the learnability, memorability and usefulness using in-person evaluation sessions. 

To measure these properties of understanding and usability of the data physicalization, questions were adopted and rewritten from the Technology Acceptance Model (TAM), Software Usability Measurement Inventory (SUMI), and System Usability Scale (SUS) to gather insight into perceived usefulness, attitude towards use, and system usability \cite{davis_perceived_1989, brooke_sus_1996}. These questions served as a baseline and starting point but were revised and adapted to suit the specifics of evaluating data physicalizations within the context of this research (see \hyperref[appendix:evaluation]{Appendix \ref*{appendix:evaluation}}).

\subsubsection{Hypothesis elicitation}

Based on the research question and creation of the prototype four hypotheses ("h" for "hypothesis") were formulated, which also acted as evaluation criteria to evaluate in the evaluation session as an observational study:

\begin{enumerate}
    \renewcommand{\labelenumi}{H\arabic{enumi}:}
    \item \textbf{Understanding:} users will exhibit a clear comprehension of the physicalization's representation of IAQ data, as well as an understanding of its intended function and impact (ease of learning)
    \item \textbf{Engagement:} users will see the utility of the design and aesthetics of the physicalization and exhibit feedback on the quality of the design.
    \item \textbf{Effectiveness:} users acceptance and satisfaction with the IAQ physicalization will be positively correlated with perceived usefulness, attitude towards use, and system usability.
\end{enumerate}

\subsubsection{Participant sampling}

The sample size of evaluation interviews was five ($n$=5) and accepted because the study findings reached theoretical saturation, new interviews did not yield new insights after the first four interviews which led to repetitive data \cite{steph_menken_introduction_2016}. Participants were gathered through purposive sampling and recruited through internal communication of the studies via email and internal lab messaging tools. The participants were not involved in the development of the prototype and had no prior knowledge of the purpose of the prototype and its design. Participants were not (financially) incentivized to take part in the studies. Respondents had to meet the inclusion criteria ("c" for "criteria") that needed to be checked before the interviews: 

\begin{enumerate}
    \renewcommand{\labelenumi}{C\arabic{enumi}:}
    \item The participant needed to use a meeting room with the building a minimum of once a week
    \item The participant needed to work within the case study week a minimum of 2 days per week
\end{enumerate}

These criteria ensured that participants were familiar with the interior design of the building and meeting rooms, allowing them to provide contextually relevant feedback. This facilitates more granular follow-up questions and provides context for discussing the prototype. This resulted in a sample of 5 participants ($n$=5) consisting of various roles and education levels (such as Software Engineering, Multimedia Design and Web Development). Participants used the meeting rooms on average one ($f$=1) time per week and worked within the lab42 building on average three ($n$=3) ($min$=1, $max$=5, $mdn$=3) days a week. On average, an evaluation session took a total of 50 minutes. The evaluation session where held between May 24th and 14th of June 2024. 

\subsubsection{Evaluation interviews}

Within the meeting room (sitting down) in the presence of the developed prototyped pre-arranged (installed in situ), semi-structured individual qualitative interviews were conducted with open-ended and nonleading questions with additional in-depth questions (follow-up probing strategy) on topics emerging from the dialogue (see \hyperref[appendix:evaluation]{Appendix \ref*{appendix:evaluation}}). To improve the quality of the interviews (and evaluation session in general) on the initial version, feedback was requested, after which the interview protocol was piloted before continuing with participants (see \hyperref[appendix:experts]{Appendix \ref*{appendix:experts}}). 

The goal was to gather first impressions and gain insight into how occupants understand the communicated data factors of the prototype. The participants were not primed in advance about the concept of the prototype to explore how the building occupants understand and use the prototype and gather first impressions.

\subsubsection{Participant observation (usability testing)}

After the explorative questions, participants were encouraged to in more detail view the prototype and interact with it as a field trial (standing up), stating anything they noticed (encouraging probing strategy). Participants' usually behavior was observed within prolonged engagement in the meeting rooms, and participants were encouraged to think aloud when viewing and interacting with the prototype (see \hyperref[appendix:usability]{Appendix \ref*{appendix:usability}}). Informal leading questions were asked about improvements, design optimization, and visual changes. In this manner, insights into particular interaction elements were acquired without the explicit involvement of HCI specialists. The goal was to test the usability of the prototype, study occupants and their behavior in a natural setting, and gather insight into the self-reflective properties of the prototype. 

\subsubsection{Prototype effectiveness}

After the interviews and observations, the participants were asked to fill in a digital online form consisting of five statements rated on a Likert scale from one (strongly disagree) to five (strongly agree) to evaluate several properties of the prototype for their effectiveness (see \hyperref[appendix:effectiveness]{Appendix \ref*{appendix:effectiveness}}). During the evaluation session, the participants filled out the survey on a researcher's laptop (sitting down). The goal was to gather quantitative measurements about the usability and effectiveness of the prototype.

\subsubsection{Transcription and coding}
All interviews were anonymized and conducted in-person on-site and audio recorded with permission of the participants. The recordings were then verbatim transcribed using the built-in Microsoft 365 transcription tool \footnote{https://www.microsoft.com/nl-nl/microsoft-365} to avoid bias while note-taking. The transcribed interviews as textual data were processed via Atlas.ti \footnote{https://atlasti.com/} for explorative qualitative coding to categorize and label possible design recommendations and improvements.