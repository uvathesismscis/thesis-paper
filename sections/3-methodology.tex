\section{Methodology}
\label{sec:methodology}

To answer the research question this study uses a human-centered approach commonly found in Human-Computer Interaction studies consisting of four phases; 1) user studies to \textit{understand} the user, 2) \textit{data collection} methods and analysis of the situation as field trials, 3) \textit{ideation} and experimental design of a prototype and 4) \textit{evaluation} and usability testing of the prototype \cite{jonathan_lazar_research_2017, zimmerman_research_2007}. This mixed-methods approach (both qualitative and quantitative)  helps understand occupants' needs and informs the design and technical set-up of the prototype evaluating the effectiveness by focusing on the user's needs from the start of the study \cite{rogers_moving_2017}. For the first phase, an online questionnaire was conducted to \textit{understand} occupants' awareness of indoor air quality (see Section \ref{sec:questionnaire}), for the second phase a lab setting was created with IAQ monitors to do \textit{data collection} and gain insight into environmental data which in turn informed phase three the \textit{ideation} and creation of the prototype. In the last phase participants \textit{evaluated} the prototype and performed usability tests.

\subsection{Case study building}

This study will be conducted in association with the Digital Interactions Lab \footnote{https://uva-dilab.com/} and will utilize the recently opened Lab42 \footnote{https://lab42.uva.nl/} building at the UvA Amsterdam Science Park \footnote{https://www.amsterdamsciencepark.nl/} as its primary case study location. Lab42 is an energy-neutral, flexible, and adaptable faculty building that facilitates collaborations among students, researchers, and businesses \cite{benthem_2022}. The buildings's layout is strategically organized into different zones, each serving various functions, ranging from quiet individual work to spaces that allow for collaborative work. Lecture halls, learning rooms, and open learning spaces make up the two lower floors, with the upper four being primarily assigned to the university academic staff, meeting rooms, and external offices (see Appendix \ref{appendix:building}). The overarching interior theme in the design revolves around 'tech' and 'nature' aiming to cultivate a fresh, light, and warm comfortable ambiance. Lab42 is an example of a smart building or living lab where sensing devices are retrofitted throughout the building to automatically adjust lighting, temperature, and the focus of this research regulating air \cite{architects_lab42_2022}. This already provides a base of environmental data that can be used and extended for further analysis. Since most of the space within the building is designated as informal learning space and another large part of the building are designed as meeting rooms (see Appendix \ref{appendix:building}), working areas these functions of focussed work and collaborative meetings can be heavily influenced by reduced cognitive performance as a result of poor indoor air quality.

\subsection{Questionnaire}
\label{sec:questionnaire}

\subsection{Data collection}

\subsection{Ideation}

\subsection{Prototyping}

\subsection{Evaluation}
