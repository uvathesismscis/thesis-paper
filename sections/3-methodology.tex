\section{Methodology}
\label{sec:methodology}

To answer the research question this study uses a human-centered approach commonly found in Human-Computer Interaction studies consisting of four phases; 1) user studies to \textit{understand} the user, 2) \textit{data collection} methods and analysis of the situation as field trials, 3) \textit{ideation} and experimental design of a prototype and 4) \textit{evaluation} and usability testing of the prototype \cite{jonathan_lazar_research_2017, zimmerman_research_2007}. This mixed-methods approach (both qualitative and quantitative)  helps understand occupants' needs and informs the design and technical set-up of the prototype evaluating the effectiveness by focusing on the user's needs from the start of the study \cite{rogers_moving_2017}. For the first phase, an online questionnaire was conducted to \textit{understand} occupants' awareness of indoor air quality (see Section \ref{sec:questionnaire}), for the second phase a lab setting was created with IAQ monitors to do \textit{data collection} and gain insight into environmental data which in turn informed phase three the \textit{ideation} and creation of the prototype. In the last phase participants \textit{evaluated} the prototype and performed usability tests.

\subsection{Case study building}

This study will be conducted in association with the Digital Interactions Lab \footnote{https://uva-dilab.com/} and will utilize the recently opened Lab42 \footnote{https://lab42.uva.nl/} building at the UvA Amsterdam Science Park \footnote{https://www.amsterdamsciencepark.nl/} as its primary case study location. Lab42 is an energy-neutral, flexible, and adaptable faculty building that facilitates collaborations among students, researchers, and businesses \cite{benthem_2022}. The buildings's layout is strategically organized into different zones, each serving various functions, ranging from quiet individual work to spaces that allow for collaborative work. Lecture halls, learning rooms, and open learning spaces make up the two lower floors, with the upper four being primarily assigned to the university academic staff, meeting rooms, and external offices (see Appendix \ref{appendix:building}). The overarching interior theme in the design revolves around 'tech' and 'nature' aiming to cultivate a fresh, light, and warm comfortable ambiance. Lab42 is an example of a smart building or living lab where sensing devices are retrofitted throughout the building to automatically adjust lighting, temperature, and the focus of this research regulating air \cite{architects_lab42_2022}. This already provides a base of environmental data that can be used and extended for further analysis. Since most of the space within the building is designated as informal learning space and another large part of the building is designed as meeting rooms (see Appendix \ref{appendix:building}), working areas these functions of focussed work and collaborative meetings can be heavily influenced by reduced cognitive performance as a result of poor indoor air quality.

\subsection{Questionnaire survey}
\label{sec:questionnaire}

To understand and collect occupants' subjective awareness and satisfaction of IAQ  a survey was created to gather quantitative data within the building as a form of Post Occupancy Evaluation (POE) (see Appendix \ref{appendix:building}). 


\subsubsection{Questions}
The questions were based on two POE studies with a focus on indoor air quality \cite{silva_post-occupancy_2017, son_perceived_2023} and used standardized questions (e.g. Q-bank) and scales (e.g. Likert-scale). The survey consisted of a total of 9 questions (5 multiple choice, 3 Likert scales, 1 not mandatory open question) consisting of questions about:

\begin{enumerate}
  \item \textit{Activity and occupancy:} the rough location the occupant is within the building, how often the occupants use the building for various activities, and how they would describe the occupancy in their current space.
  \item \textit{Awareness and satisfaction:} how aware the occupant is of the current air quality in the space, how the occupant perceives the air quality in the current space, and how satisfied the occupant is with the air quality in the current space.
  \item \textit{Health and cognitive symptoms :} if the occupant experiences any health or cognitive symptons based on the air quality in the current space.
\end{enumerate}


\subsubsection{Participants}
The survey was distributed via handouts with QR Codes to occupants present at the informal learning spaces of the atrium, first floor, and second floor. Additionally, handouts were attached to the tables using stickers. All instances of participation were voluntary and conducted without remuneration. Distribution of the survey was open for participation for four weeks within the Lab42 building which recorded XX ($n$=XX) responses.

\subsubsection{Data analysis}
After the distribution of the survey completed analysis of the collected data was performed in the form of data cleanup and exploratory data visualization. In Python (Jupyter Notebook format) \footnote{https://jupyter.org/} Libraries such as Numpy \footnote{https://pandas.pydata.org/} were used to clean the data (e.g. remove non-consenting users) and visualization libraries such as Seaborn \footnote{https://seaborn.pydata.org/} were used to create graphs and plots (e.g. boxplot the likert-scales) to get an overview of the collected data and gain insight into understanding the occupants.

\subsection{Monitoring}

To gather data about the current situation of air quality within the building and understand the current situation within the building in terms of air quality data we created a lab experiment with two specific meeting rooms within the building.

\subsubsection{Technical set-up}



\subsection{Ideation}

List a library of over 300 data physicalizations. Two state of the art papers do systematic reviews of physicalization with a combined examples of around 64 projects of which both academic ($f$=34) and non-academic ($f$=17) samples. With around three ($f$=3) projects (PhysiKit, other, other \cite{sauve_physecology_2022}) extensively studied and reviewed for this research which focus in some form or way on airflow and air quality

\subsection{Prototyping}

\subsection{Evaluation}
