\section{Results}
\label{sec:results}

The goal of the chosen methodologies (see \hyperref[sec:related_work]{Section \ref*{sec:related_work}}) allows for answering of the \hyperref[rq:1]{(sub)research questions}. First, the results of the questionnaire survey are described (see \hyperref[sec:survey_analysis]{Section \ref*{sec:survey_analysis}}) and key insights into the air quality monitor data collection are presented (see \hyperref[sec:monitor_analysis] {Section \ref*{sec:monitor_analysis}}) to answer \hyperref[subq:1]{SQ\ref*{subq:1}}. Then the developed prototype will be presented in section (see \hyperref[sec:prototype_results]{Section \ref*{sec:prototype_results}}) which addresses \hyperref[subq:2]{SQ\ref*{subq:2}} and the results from the evaluation sessions will be summarized in (see \hyperref[sec:evaluation_results]{Section \ref*{sec:evaluation_results}}) to address \hyperref[subq:3]{SQ\ref*{subq:3}}.

\subsection{Survey analysis}
\label{sec:survey_analysis}

The following section presents the outcomes of the survey conducted to assess various aspects related to occupants' activity, IAQ awareness, satisfaction, perception, and health impacts among Lab42 building occupants. The responses show that most occupants are located in the atrium of the building, not particularly aware of the IAQ, and perceive the IAQ as acceptable based on the large open area and planters, meeting satisfactory IAQ levels. 

\subsubsection{Activity and occupancy}

The first set of questions of the survey focussed on activity and occupancy. On average, occupants who filled in the survey used the Lab42 building \textit{3 times a week} for various activities ($f$=X). As opposed to 1 day a week, 4 days a week, 2 days a week with a small number of occupants using the building 5 days a week. Most occupants from the sample size were located on the \textit{ground floor ($f$=X) and the first floor ($f$=X)} as opposed to the 2nd floor. Both the first floor and ground floor are considered the 'open area' and part of the atrium of the building. This is by design of the building where the lower floors have more co-working spaces to be used as informal open learning spaces whereas the upper floors are more designed as meeting rooms and private working offices. Many occupants described the overall occupancy in the space as \textit{not too crowded} ($f$=X) with a smaller percentage explicitly stating the space was crowded or not crowded at all.

\subsubsection{Awareness, perceived and satisfaction}

The second set of questions were three Likert scales about perceived IAQ, awareness of IAQ, and satisfaction with IAQ. Over half of the occupants are \textit{not particularly aware} of the IAQ in their current space, either very unaware ($f$=X), unaware ($f$=X), or neutral ($f$=X) the minority of the occupants are aware or very aware. Indicating that in general occupants are not very aware of the IAQ if they are directly asked about it. However, once occupants are asked about their perceived IAQ and satisfaction the majority of the occupants perceive the IAQ as \textit{acceptable ($f$=X) or good ($f$=X)} and almost all occupants consider the IAQ as satisfactory by being \textit{satisfied ($f$=X) or very satisfied ($f$=X)} with the IAQ.

\subsubsection{Health and cognitive symptons}

The third set of questions were to indicate if users suffered from health or cognitive symptons based on the IAQ. The majority of participants answered \textit{none} to both the health ($f$=X) and cognitive ($f$=X) questions. A small percentage experienced health symptoms such as headaches and feeling noiziating and a larger percentage experienced cognitive symptons such as trouble with focus or tiredness. The results of this section of the survey are inconclusive since it's difficult to determine if these symptons are specifically related to the air quality.

\subsubsection{Open-ended air quality description}
The most notable finding of the open question is that the occupants who filled in the not mandatory question describing their perception of IAQ mention specifically the openness of the atrium space:

\begin{quote}
P11: "[...] think it is good, [...] although I must say that this is mostly based on the large amount of open space in the building"
\end{quote}

With some of the occupants mentioning the 'high ceilings' of the atrium specifically.

\begin{quote}
P8: " [...] feel like in this building the air quality is really good, mainly because of the impression the high ceilings give"
\end{quote}

\begin{quote}
P13: "I like the air quality. This may also be because I sit close to the door and the ceiling is high."
\end{quote}

Another notable attribute is that many occupants describe contributing to the perception that the IAQ is sufficient are the hanging planters and greenery that is present within the atrium.

\begin{quote}
P16: "[..] I also see green plants around me, of which I think they are real."
\end{quote}

\begin{quote}
P21: "I thrive on the oxygen provided by the many plants here [..]"
\end{quote}

\begin{quote}
P14: "[...] and the hanging plants that are present.
\end{quote}



\subsection{Air Quality Monitors}
\label{sec:monitor_analysis}

Show graphs of the sample data. Hypothesis is that when meetings occur with more occupants after a while the air quality exceeds recommended standards.

\subsection{Prototype}
\label{sec:prototype_results}

Based on the requirements, data physicalization design principles, and concept models exploration a final high fidelity (hi-fi) version of the prototype was developed that functioned as a proof-of-concept of the physical design solution as a feasibility study and utilized in the user study for evaluation.

\subsubsection{Concept description}

\textit{Bluebird} is a hanging kinetic type sculpture inspired by organic nature materials and the shapes of hanging planters that encode the environmental properties of indoor air quality data. It is meant to be hung from the ceiling in small to medium rooms and changes based on real-time air quality monitor data. Strings (plant branches) either become longer or smaller simulating the growth of a plant. Movements of the leaves indicate the freshness of air and movement. The overall design philosophy of the shapes and forms uses the notion of calm technology to minimize interruption cost \cite{case_calm_2016}. 

\subsubsection{Electronics and components}

A controller device running on an Arduino Uno R3 \footnote{https://store.arduino.cc/products/arduino-uno-rev3} microcontroller with an MKR Motor shield is used \footnote{https://store.arduino.cc/products/arduino-motor-shield-rev3} to control six 360° MG90S type Micro Servo Motors \footnote{https://www.towerpro.com.tw/product/mg90s-3/}. Attached to these motors are pulleys with fishing lines simulating the growth of the hanging planter so that the string can be moved up and down.


\subsubsection{Crafting technologies and materials}
The strings, leaves, and housings of the electronics and mechanical hardware are created using additive manufacturing (3D Printing) using a Fused deposition Modeling (FDM) technique using Polylactic acid (PLA) plastic filament in various colors. The electronics enclosures and plant models were modeled using computer-aided design (CAD) software. A digital fabrication technique commonly found in data physicalization prototypes \cite{anhalt_university_germany_design_2022}. To create leaves representing textile or fabric custom properties were defined within the 3D printing software (Slicing) to remove top and bottom layers and create a thin layer of infill.

\subsubsection{System Architecture and software}

The microcontroller uses custom firmware written in Arduino code \footnote{https://www.arduino.cc/reference/en/} (similar to C++) that receives real-time data from the air quality monitors using the LoRaWAN \footnote{https://lora-alliance.org/about-lorawan/} communication protocol to control the mechanics of the prototype (see \hyperref[appendix:architecture]{Appendix \ref*{appendix:architecture}}). This arrangement of hardware is commonly found in Internet of Things (IoT) architecture set-ups and follows the notion of Edge Computing with a (1) sensing, (2) networking, (3) processing, and (4) application layer \cite{li_edge-oriented_2019, idrees_edge_2018}.

\subsubsection{Experimental set-up}

The prototype was hang-up at the two meeting rooms. Interaction set-up etc. Write about how the prototype is in the room. Installed (reference lab settings again). Reference figures in appendix for more photographs and impressions.

We derived five relevant dimensions based on the literature; \textit{audience, intention, interaction, philosophy, representation} \cite{sauve_physecology_2022, hornecker_design_2023}.

\subsubsection{Audience}

\subsubsection{Intention}

\subsubsection{Interaction}

\subsubsection{Data representation}

Write about data scale (stevens) nominal, ordinal and numerical. Needs electronic components (e.g. microcontrollers, sensors) and non-electronic components.

\subsubsection{Design elements}

\begin{itemize}
  \item \textbf{Sight}: visually see string become largers. Acts as a metaphor of plant 'growth'. The better the air quality the more the plant can 'grow'.
  \item \textbf{Movement}: if fresh air comes in we indicate this through movement as a metaphor for wind gusses between a field of grass or plants.
\end{itemize}

\subsubsection{Experimental set-up}

Write about how the prototype is in the room. Installed (reference lab settings again). Reference figures in appendix for more photographs and impressions.

\subsubsection{Haptics}

Write about interaction and haptics it support.

\begin{itemize}
  \item \textbf{Sight}: visually see string become largers. Acts as a metaphor of plant 'growth'. The better the air quality the more the plant can 'grow'.
  \item \textbf{Movement}: if fresh air comes in we indicate this through movement as a metaphor for wind gusses between a field of grass or plants.
\end{itemize}

\subsection{Evaluation}
\label{sec:evaluation_results}

The evaluation methods (see \ref{sec:questionnaire}) are used to assess the quality and impact of the data physicalisation with a focus on how users perceive the data embedded in the physical representation and what long-term impact it has on people.


\subsubsection{Understanding (learnability)}

Did users understand the phys? Know what it was visualizing? Users reaction?

\subsubsection{Self-reflection (memorability)}

Makes you more aware of IAQ?

\subsubsection{Effectiveness (efficiency)}

Suspected to change habits based on output? Attitude change? Behavioral stimulation.



% Sometimes,  especially  if  you  have  quite  different experiments or research  questions,  it makes sense to interleave the experimental setup and the results sections, so the reader does not get lost. It is then helpful to structure clearly in (sub)subsections.