\section{Results}
\label{sec:results}

The goal of the chosen methodologies (see \ref{sec:questionnaire}) where choses to answer the research questions. Section (see \ref{sec:questionnaire}) describes the results of the survey and section (see \ref{sec:questionnaire}) shows key insights into data collection which address SQ1. Then the developed prototype will be presented in section (see \ref{sec:questionnaire} which addresses SQ2 and the results from the evaluation will be summarized in (see \ref{sec:questionnaire}) to address SQ3.

\subsection{Survey analysis}
Show the results of the survey, preferably with graphs based on the likert-scale. Hypothesis is that awareness of indoor air quality among occupants is very low. This is comparitive to other studies performed that indicate the same results 

\subsection{Air Quality Monitors}

Show graphs of the sample data. Hypothesis is that when meetings occur with more occupants after a while the air quality exceeds recommended standards.

\subsection{Prototype}

Based on the ideation and exploratory prototyping a final version of the prototype was created.

\subsubsection{Hardware design}

About materials and crafting technologies (laser-cutting), 3d printing etc.

\subsubsection{System Architecture}

Write about software etc. (lorawan, ) Write about layers and edge computing. 

\subsection{Prototype Evaluation}

The evaluation methods (see \ref{sec:questionnaire}) are used to assess the quality and impact of the data physicalisation with a focus on how users perceive the data embedded in the physical representation and what long-term impact it has on people.

\subsubsection{Understanding (qualitative)}

Did users understand the phys? Know what it was visualizing? Users reaction?

\subsubsection{Self-reflection}

Makes you more aware of IAQ?

\subsubsection{Effectiveness (question answering)}

Suspected to change habits based on output? Attitude change? Behavioral stimulation.



% Sometimes,  especially  if  you  have  quite  different experiments or research  questions,  it makes sense to interleave the experimental setup and the results sections, so the reader does not get lost. It is then helpful to structure clearly in (sub)subsections.