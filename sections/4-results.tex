\section{Results}
\label{sec:results}

The goal of the chosen methodologies (see \hyperref[sec:analysis]{Section \ref*{sec:analysis}}) where choses to answer the \hyperref[rq:1]{(sub)research questions}. Section (see \hyperref[sec:analysis]{Section \ref*{sec:analysis}}) describes the results of the survey and section (see \hyperref[sec:analysis]{Section \ref*{sec:analysis}}) shows key insights into data collection which address \hyperref[subq:1]{SQ\ref*{subq:1}}. Then the developed prototype will be presented in section (see \hyperref[sec:analysis]{Section \ref*{sec:analysis}}) which addresses \hyperref[subq:2]{SQ\ref*{subq:2}} and the results from the evaluation will be summarized in (see \hyperref[sec:analysis]{Section \ref*{sec:analysis}}) to address \hyperref[subq:3]{SQ\ref*{subq:3}}.

\subsection{Survey analysis}
Show the results of the survey, preferably with graphs based on the likert-scale. Hypothesis is that awareness of indoor air quality among occupants is very low. This is comparitive to other studies performed that indicate the same results 

\subsection{Air Quality Monitors}

Show graphs of the sample data. Hypothesis is that when meetings occur with more occupants after a while the air quality exceeds recommended standards.

\subsection{Prototype}

The prototype was hang-up at the two meeting rooms. Interaction set-up etc.

We derived five relevant dimensions based on the literature; \textit{audience, intention, interaction, philosophy, representation} \cite{sauve_physecology_2022, hornecker_design_2023}.

\subsubsection{Audience}

\subsubsection{Intention}

\subsubsection{Interaction}

\subsubsection{Design elements}

\begin{itemize}
  \item \textbf{Sight}: visually see string become largers. Acts as a metaphor of plant 'growth'. The better the air quality the more the plant can 'grow'.
  \item \textbf{Movement}: if fresh air comes in we indicate this through movement as a metaphor for wind gusses between a field of grass or plants.
\end{itemize}

\subsubsection{Experimental set-up}

Write about how the prototype is in the room. Installed (reference lab settings again). Reference figures in appendix for more photographs and impressions.

\subsubsection{Haptics}

Write about interaction and haptics it support.

\begin{itemize}
  \item \textbf{Sight}: visually see string become largers. Acts as a metaphor of plant 'growth'. The better the air quality the more the plant can 'grow'.
  \item \textbf{Movement}: if fresh air comes in we indicate this through movement as a metaphor for wind gusses between a field of grass or plants.
\end{itemize}

\subsection{Evaluation}

The evaluation methods (see \ref{sec:questionnaire}) are used to assess the quality and impact of the data physicalisation with a focus on how users perceive the data embedded in the physical representation and what long-term impact it has on people.


\subsubsection{Understanding (qualitative)}

Did users understand the phys? Know what it was visualizing? Users reaction?

\subsubsection{Self-reflection}

Makes you more aware of IAQ?

\subsubsection{Effectiveness (question answering)}

Suspected to change habits based on output? Attitude change? Behavioral stimulation.



% Sometimes,  especially  if  you  have  quite  different experiments or research  questions,  it makes sense to interleave the experimental setup and the results sections, so the reader does not get lost. It is then helpful to structure clearly in (sub)subsections.