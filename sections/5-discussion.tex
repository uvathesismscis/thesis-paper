\section{Discussion}
\label{sec:discussion}
% Compare your results with the state-of-the-art and reflect upon the results and limitations of the study. You can already hint at future work to which you come back in the conclusion section.

\subsection{Ethical considerations}

\subsection{Limitations}

\subsubsection{Validity}


\subsubsection{Prototype scalability}
Concessions were made during the creation of the prototype. Limitations in time, hardware availability and technical limitations. Design improvements are prevelent with the design of the prototype. The look and feel of a prototype can impact the effectiveness of the design as users perceive the design as a non-functioning prototyping.

\subsubsection{Generalizability}
Tested within the Lab42 building, so it's hard to test effectiveness are scalable to other (university) buildings. There are many environmental properties specific to this building (furniture, room allocation) that gives this context. There is inherent bias in the user group, although distribution of the surveys and interviews were voluntary and random (in terms of gender, role) geographics and location play a role. The case study building is a university building so it is expected that the data is biased towards higher-educated university students and staff members and not representative of a larger population. It is not clear if the results can be externally validated and generalized from sample size to larger population.

\subsection{Future work}

Future work, the prototype to communicate more sensory properties. We envision fabric that holds water to communicate humidity etc. made of organic materials etc.