\section{Discussion}
\label{sec:discussion}
An examination of the grounded theory research design and methodologies utilized in this study combined with an interpretation of both qualitative and quantitative findings reveal several comparative findings, study limitations, and opportunities for future research.

\subsection{Findings in context}

Comparing the findings of the existing literature (see \hyperref[sec:related_work]{Section \ref*{sec:related_work}}) the results of this study (see \hyperref[sec:results]{Section \ref*{sec:results}}) shows that general findings from the literature are supported. A key distinction lies in this study's human-centered design approach, which incorporated occupants' feedback throughout the iterative research process. The findings from one methodology informed decisions on subsequent approaches in the study (see \hyperref[sec:methodology]{Section \ref*{sec:methodology}}). Furthermore, it is important to note that these comparative studies vary significantly in their context (e.g., outdoor environments) and lab settings (e.g., households).

\subsubsection{Occupant awareness and IAQ monitor}
Occupant awareness of indoor air quality (IAQ) is generally low (\hyperref[subq:1]{SQ\ref*{subq:1}}) due to its inherent properties, making it difficult to detect through human sensory perception. Occupants often judge indoor air quality based on visible attributes within the environment, which usually do not positively influence air quality \cite{schweizer_indoor_2007} \cite{corlan_importance_2021}. Building activity, occupancy, and crowdedness significantly impacts the accumulation of CO$_{2}$ concentrations (\hyperref[subq:2]{SQ\ref*{subq:2}}), the primary factor of suboptimal air quality in indoor environments \cite{fromme_indoor_2023} \cite{du_indoor_2020}, especially in smaller spaces like offices and meeting rooms \cite{zhong_complexity_2021}.

\subsubsection{Prototype development and evaluation}
The additional step undertaken in this study was the creation of the prototype which extends beyond the scope of previous related studies. The prototype's development employed techniques and manufacturing methods commonly used in data physicalization projects \cite{alexander_data_2019, jansen_opportunities_2015}, while its evaluation followed methodologies frequently applied in HCI research \cite{ranasinghe_encoding_2023, sauve_physecology_2022}. Evaluation sessions indicated increased understanding and self-reflection regarding data insights and revealed several design improvements for the prototype (\hyperref[subq:2]{SQ\ref*{subq:3}}). However, no significant effect or success rates were measured of the prototype's ability to help occupants take preventive action for long-term behavioral change.

\subsection{Limitations}

The data collection, design, and validation process showed several strengths but also had some limitations that should be noted. The primary limitations encountered during this study were the reproducibility of the evaluation sessions, monitoring validity prototype scalability and generalizability of findings to other buildings.

\subsubsection{Evaluation reproducibility}

Grounded theory is an established research method, but greater care and effort should have been taken for investigator and theory triangulation. Some concepts may have been missed during the ideation phase since only the main researcher was involved in the prototyping. During evaluation sessions, the lack of four-eyed principles left room for researcher and interviewee biases. The small sample size, although it reached saturation, limited generalizability, and demographic diversity. A more varied interview setting, such as different-sized meeting rooms or personal office spaces, could have tested usability in diverse contexts. The same group of participants was used for both interviews and the user study, potentially biasing their evaluations. And while this study used similar approaches to others, including HCI evaluation and SUS metrics, the methods were slightly adapted, impacting the reproducibility and the validity of standardized quantitative results. 

\subsubsection{Prototype scalability}
The prototype used in user studies is a proof of concept but requires further optimization. Concessions were made due to limitations in hardware availability and technical data constraints. Design improvements are prevalent, acknowledging that there is no universally perfect way to display air quality data. The prototype's current appearance can influence its effectiveness as users might perceive it as non-finished. A significant limitation is the inability of the current setup to explore long-term intervention strategies due to the need for testing over an extended period. 

\subsubsection{Building generalizability}
The methodologies employed within Lab42 and its specific context may limit generalizability to other buildings and environments with different characteristics.  The building's unique environmental properties (furniture, room allocation) contribute to this context. This also means there is an inherent bias in the user group, despite voluntary and random survey and interview distribution, convenience sampling for participant recruitment may introduce a bias toward higher-educated university students and staff members present in specific areas of the building, causing a higher attrition rate, which may not be representative of a larger population. 

\subsubsection{Data collection validity}
Care was taken using standardized high-quality sensors and the well-defined data collection procedures ensured reliability, monitoring within meeting rooms primarily focused on environmental properties like CO$_{2}$ concentrations, lacking explicit data on occupancy. Adding more context to interpret occupant behavior and CO$_{2}$ concentration patterns, as well as considering the effects of existing mechanical ventilation could enhance the data collection phase.

\subsection{Future work}

Future research should prioritize enhancing the design and communicative properties of the prototype, along with implementing a more structured lab environment to gather quantifiable results on its effectiveness and its support in long-term preventive action among occupants. 

\subsubsection{Design enhancements}
Enhancements to the prototype include communicating additional sensory properties, such as conveying humidity through organic materials like fabric and color-coding leaves to better illustrate predictions and enhance the perception of fresh air. Incorporating more material-driven feedback aims to deepen engagement and foster embodied, multi-sensory experiences.

\subsubsection{Long-term effectiveness}
Evaluating the prototype's efficacy beyond initial impressions requires addressing validity and control measures for concrete effectiveness assessments, possibly through broader behavioral data collection in daily life scenarios, possibly by incorporating window sensors to track occupants' window usage during meetings. To ensure the generalizability of findings, replication across diverse environments is essential. Replicating the study in multiple buildings with diverse occupants would strengthen the generalizability of results, given the unique characteristics of Lab42, including its architecture and occupant demographics.