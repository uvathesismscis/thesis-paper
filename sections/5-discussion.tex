\section{Discussion}
\label{sec:discussion}

Analyzing the grounded-theory research design and chosen methodologies of this study, along with interpreting the qualitative and quantitative findings reveals several comparative findings, study limitations and possibilities for future research.


\subsection{Findings in context}
Comparing the findings of the existing literature (see \hyperref[sec:related_work]{Section \ref*{sec:related_work}}) with the results of the study of this paper (see \hyperref[sec:results]{Section \ref*{sec:results}}), general findings were supported. 

\subsubsection{Occupant awareness and IAQ monitor}
Occupant awareness about IAQ is generally low (\hyperref[subq:1]{SQ\ref*{subq:1}}) based on its inherent properties of not being easily detected by human sensory reception and the perception of the quality of indoor air by occupants is judged mainly by visible attributes within the indoor environment that in most cases don't influence the air quality in a positive way \cite{schweizer_indoor_2007} \cite{corlan_importance_2021}. Building activity, occupancy and crowdedness of a space largely have a negative impact on CO2 concentrations (\hyperref[subq:2]{SQ\ref*{subq:2}}) which is the main factor of suboptimal air quality within indoor environments. \cite{fromme_indoor_2023} \cite{du_indoor_2020} specifically within smaller spaces such as offices and meeting rooms \cite{zhong_complexity_2021}. 

\subsubsection{Prototype development and evaluation}
Due to the explorative nature of the research and the novelty of the data physicalization prototype (\hyperref[subq:3]{SQ\ref*{subq:3}}), it's difficult to compare these results directly and quantitatively to existing literature. However, the evaluation of the prototype uses a similar approach to other existing data physicalization studies \cite{alexander_data_2019, jansen_opportunities_2015}. The evaluation sessions show an increase in understanding and self-reflection of data insights but no significant effect or success rate is observed if the data physicalization helps occupants take preventive action. 

\subsubsection{Key differences from comparitive studies}

A key difference however is the human-centered design approach where feedback from occupants is incorporated throughout the iterative research process. The findings of one methodology informed decisions on how to further approach the study (see \hyperref[osec:methodology]{Section \ref*{sec:methodology}}). Additionally, it should be noted that these comparative state-of-the-art studies also heavily vary in their context (e.g. outdoor environments, non-university buildings) and lab setting (e.g. households, classrooms) which differ from the Lab42 building. The additional step undertaken in this study was the creation of the prototype which goes beyond what has been observed in previous related studies.

\subsection{Limitations}

The design and validation process showed several strengths but also had some limitations that should be noted. The main limitations encountered during this study were the scalability and reliability of the prototype, the generalizability of findings to other buildings, and the adoption of standardized questionnaires.

\subsubsection{Evaluation validity}

Grounded theory is a well-established and widely recognized research method. However during the research more care and effort should have been taken for investigator and theory triangolution. Some concepts in the ideation phase possibly did not emerge, as only the main researcher was involved in the prototyping phase. During the evaluation sessions no other four-eyed principles were used to mitigate any researcher (confirmation) and interviewee bias leaving the interpretative nature of qualitative research. The evaluation sessions were done with a rather small sample size since interviews reached saturation but the methodology could have benefitted from a larger sample size for better generalizability and a more diverse demographic but mainly a more diverse interview setting (e.g. different sized meeting rooms, personal office spaces) to test the usability within different contexts. The homogenous setting the prototype was evaluated in may have influenced the results and limits the understanding of its effectiveness for a more diverse group. The same group of participants was used for both the interviews and the user study, which may have influenced their evaluation of the product during the user study. The initial interview responses could have biased their perceptions when assessing the product’s effectiveness.

\subsubsection{Prototype scalability}
The prototype used for user studies is a proof of concept and should be further optimized. Concessions were made during the creation of the prototype. Limitations in time, hardware availability and technical limitations. Design improvements are prevelent with the design of the prototype and acknowledge that there is no universally perfect form of displaying air quality data that satisfies all users. The look and feel of a prototype can impact the effectiveness of the design as users perceive the design as a non-functioning prototyping. Usability tests and observations were done in natural settings which means participants have different contextual understandings of situations which can skew the interperation of the prototype within the lab setting (meaning in context). The current research setup cannot discover long-term intervention strategies of the prototype cause it requires testing over a longer period. During the prototype testing scenarios of high CO2 concentrations were simulated but the accuracy of the findings depend on actual real-life scenarios of which these situations occur. Together with the first limitation, this is the reason why only positive indications of the prototype are found in this study, and no hard conclusions can be drawn on it's effectiveness. 

\subsubsection{Building generalizability}
Tested within the Lab42 building, so it's hard to test effectiveness are scalable to other (university) buildings. There are many environmental properties specific to this building (furniture, room allocation) that gives this context. There is inherent bias in the user group, although distribution of the surveys and interviews were voluntary and random (in terms of gender, role) geographics and location play a role. The case study building is a university building so it is expected that the data is biased towards higher-educated university students and staff members causing a high attrition rate and may not be representative of a larger population. The convenience sampling approach used for participant recruitment may introduce sampling bias, as it primarily targets occupants present in specific areas of the building.It is not clear if the results can be externally validated and generalized from sample size to larger population. The specific context of Lab42 may limit generalizability to other buildings or environments with different characteristics. 

\subsubsection{Standardized questions reliability}

This study uses similar approaches to the other studies in terms of utilizing the same dimensions and metrics (e.g. HCI evaluation, POE evaluation, standardized questionnaire questions) wherever possible but these methods have been slightly adopted and tweaked for this specific research influencing the reproducibility and validity of the results and gathering standardized quantitative calculations (Cron Alpha score e.g.).

\subsubsection{Data collection}

Care was taken to ensuring consistency in data collection using standardized high-quality sensors and well-defined procedures for data collection ensured reliable and consistent data it needs to be noted that he monitoring within the meeting rooms mainly focussed on envirnomental properties such as CO2 concentrations. No explicit data was gathered about how many occupants where in each meeting etc. The data collection phase could benefit from more context to intepret occupant behaviour and explain CO2 concentration patterns in more details as well as the effects of the mechanical ventilation already existing in the building.

\subsection{Future work}

Future research should focus on improving the design and communicative properties of the prototype and a more structured lab environment to gather more quantifiable results on the effectiveness of the prototype and helping occupants take preventive action long-term. E.g. installing window sensors to measure how often occupants in meetings open the windows. The absence of objective outcome measures limits the validity of conclusions regarding the prototype's effectiveness. Incorporating robust validity checks would enhance the credibility of the evaluation findings. 

The current focus of the prototype is in terms of awareness to occupants and wellbeing purposes and not as a behavior-change intervention. Further research with a larger and more diverse sample is needed to establish its applicability. The evaluation of the prototype focusses on understanding and first impressions provide a framework for assessing prototype efficacy, the evaluation session lack validity measures and control measures for concrete effectiveness measurements. Broader behavioral data, such as how the intervention influenced participants' actions and behavior in daily life, need to be addressed in future deployment studies.


Extending the duration of product usage in future studies could yield more accurate results. We envision the prototype to communicate more sensory properties. We envision fabric that holds water to communicate humidity etc. made of organic materials etc. The prototype could have more material-driven feedback to further engage in more embodied experiences.

Without replication in diverse environments, the extent to which findings can be generalized to broader populations or settings is uncertain. Replicating the study across multiple buildings with diverse occupants would strengthen the generalizability of the findings. The unique characteristics of Lab42, such as its architecture and occupant demographics, may not be representative of other indoor environments. 