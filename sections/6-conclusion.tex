\section{Conclusion}
\label{sec:conclusion}

The limited research on occupants' subjective comfort needs and the potential of visualizing IAQ data for building occupants is addressed in this study. Using a mixed-methods approach, it examines occupants' awareness of indoor air quality, correlating this with air quality monitoring data, and explores how physically communicating these properties can aid in increasing data insights and serve as an intervention strategy. The findings indicate that occupants are generally unaware of indoor air quality, often judging it based on visual aspects like open spaces and planters. Monitoring CO2 levels in meeting rooms shows that concentrations frequently reach suboptimal thresholds, posing health and cognitive risks.  The results of the developed prototype and evaluation of the data physicalization prototype suggest a positive attitude toward interpreting IAQ data and taking preventive actions, although these are preliminary indications without statistically significant results. 

The limitations around the generalizability of the specific context of the case study university building and the sample size of the evaluation sessions should be considered when interpreting the conclusions and applying the findings to other contexts. Future research should aim to verify these findings in more diverse settings and further develop the design and aesthetics of the prototype to include even more environmental properties for multi-sensory interaction assuming it increases the effectiveness of the data physicalization. It also lays the groundwork for further research within the niche of data physicalization, addressing the research gap in understanding occupants' behavior to take preventive actions based on their indoor environment. Ultimately, this study contributes to the field of Human-Building Interaction by integrating computing technologies into built environments, aiding building staff and architecture in making decisions about structuring spaces to improve occupants' well-being.
