\section{Conclusion}
\label{sec:conclusion}

The limited research on occupants' subjective comfort needs and the potential of visualizing IAQ data for building occupants is addressed in this study. Using a mixed-methods approach, it examines occupants' awareness of IAQ, correlates this with air quality monitoring data, and explores how physically communicating these properties can increase data insights and serve as an intervention strategy. Findings reveal occupants' general unawareness of IAQ, often judging it based on visual aspects of the indoor space. Monitoring CO2 levels in meeting rooms indicate frequent suboptimal thresholds, posing health and cognitive risks. The developed prototype and its evaluation suggests a positive attitude toward interpreting IAQ data and taking preventive actions, albeit with preliminary indications lacking statistical significance and long-term effectiveness results. 

Limitations include the specific context of the university building affecting generalizability and the methods used to evaluate the prototype limiting the reproducibility and quantifiable metrics. Future research should verify findings in diverse settings with a more rigorous long-term lab environment and enhance the prototype's design and aesthetics for multi-sensory interaction, possibly enhancing the effectiveness of its physicalization properties. Ultimately this study establishes a foundation for future research within the niche of data physicalization addressing the research gap in understanding occupants' behavior to take preventive actions based on their indoor environment.  It contributes to Human-Building Interaction by integrating computing technologies into built environments, assisting building staff and responsive architecture in optimizing space design to enhance occupants' overall comfort.

\newpage